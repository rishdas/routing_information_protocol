
\documentclass[10pt]{extarticle}
\usepackage[utf8]{inputenc}     % permite edição direta com acentos
\usepackage[pdftex]{hyperref}
\usepackage[margin=0.5in]{geometry}
\usepackage{fancyvrb}
\usepackage{graphicx}

\hypersetup{
  pdfauthor={Lucas Brasilino and Sirshak Das},
  pdftitle={Computer Network Assignment 3 Write-up},
  pdfsubject={Computer Network},
  pdfkeywords={{Lucas Brasilino}, {Sirshak Das}},
  colorlinks=true,
  urlcolor=blue,
  linkcolor=black
}

\title{Computer Network Assignment 3 Write-up}
\author{Lucas Brasilino \\ 
  \href{mailto:lbrasili@indiana.edu}{lbrasili@indiana.edu}
  \and Sirshak Das\footnote{Names placed in alphabetical order} \\  
  \href{mailto:sirdas@indiana.edu}{sirdas@indiana.edu}
}
\begin{document}                                 
\thispagestyle{empty}
\maketitle
%\centerline{\Large\textbf{Computer Network Assignment 3 Write-up}}   
%\vspace{0.1in}
%\centerline{\large{Lucas Brasilino and Sirshak Das}\footnote{Names placed in
%    alphabetical order}}
%\centerline{\texttt{\href{mailto:lbrasili@indiana.edu}{lbrasili@indiana.edu}}}
%\vspace{0.01in}

\section{Introduction}

The third programming assignment from P358 Computer Networks class is to
implement a Distance Vector routing. Each node, which are routers, has to send
their cost for reaching all nodes within the network to its immediate
neighbors. 

The resulting Routing Information Protocol (RIP) service from this assignment
might support update messages. These messages might be sent in a periodic and
asynchronous\footnote{As soon as there is some routing table update} ways. The
protocol is well defined in Figure~\ref{fig:protocol} and must use UDP. After
receiving update messages, a node uses \href{https://en.wikipedia.org/wiki/Bellman%E2%80%93Ford_algorithm}{Bellman-Ford Algorithm} to update
its routing table. Finally, if a node updates any costs in its table, it sends
an update message to its immediate neighbors.

\begin{figure}[!h]
\centering
\begin{BVerbatim}

                           32 bits
+--------------------------------------------------------------+
|                       IP address (4 bytes)                   |
+--------------------------------------------------------------+
|                          cost (4 bytes)                      |
+--------------------------------------------------------------+ 
|                       IP address (4 bytes)                   |
+--------------------------------------------------------------+
|                          cost (4 bytes)                      |
+--------------------------------------------------------------+
                             ...
\end{BVerbatim}
\caption{Update message}
\label{fig:protocol}
\end{figure}

\section{Implementation}

The developed program, called \texttt{rip}, was written in C using a private
repository in IU's GitHub server. It is organized in 3 directories:

\begin{itemize}
\item{\texttt{src}:} stores the source code files (\texttt{*.c})
\item{\texttt{include}:} stores the header files (\texttt{*.h})
\item{\texttt{test}:} stores sets of configuration files (\texttt{*.conf})
\end{itemize}

Some design decisions were made for \texttt{rip} development. First, it is a
multithreaded application. Second, internally not only node's IP addresses are
stored and manipulated, but also the node's hostname. This fact enables more
flexibility on the configuration file and better routing table
printouts. Finally, the functions in source code follows the convention
\texttt{rip\_<file>\_<name>()}, are organized in such a way that 
functions starting with \texttt{rip\_<file>\_*()} are placed in \texttt{rip\_<file>.c}
and each filename embedds its purpose, as described bellow:
\begin{itemize}
\item{\texttt{rip\_main.c}:} It is the application entry point (function
  \texttt{main()}) and the main thread. Also contains routines for parsing the
  command line arguments and configuration file, for the distance vector
  graph initialization and for running the mainloop. 
\item{\texttt{rip\_net.c}:} Encompass all network API, including the
  \textbf{send\_advertisement} component
  (\texttt{rip\_net\_send\_advertisement()}).
\item{\texttt{rip\_obj.c}:} This file contains all memory allocation
  deallocation functions for creating objects, like node information, route
  table entries, and protocol message entries.
\item{\texttt{rip\_routing.c}:} This file contains all the routing APIs including filling up of graph structure, distance vector and routing table.   
\item{\texttt{rip\_up.c}:} Contains the entry points for the other two threads:
  \texttt{rip\_up()} and \texttt{rip\_up\_ttl()}.
\end{itemize}

\section{Compilation}

On top directory a \texttt{Makefile} is provided for a ease compilation. As
expected, packets like gcc, make and glibc-dev must be installed in the Linux
system.

\begin{Verbatim}
$ make clean && make
\end{Verbatim} %$

At the end of compilation process, the \texttt{rip} binary will be available in
this directory.

\section{Running the code}




\section{Evaluation}


\section{Documentation of the code}


\end{document}

